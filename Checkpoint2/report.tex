\documentclass[12pt]{article}
\usepackage{amsmath, amssymb, graphicx, hyperref}
\usepackage{geometry}
\geometry{margin=1in}

\title{\textbf{Check-Point 2 Report} \\ Analysis and Implementation Outlook of Edge-Connectivity Augmentation}
\author{Roshaan Khan - rk08103 \\ Hamza Ansari - ha08033}
\date{\today}

\begin{document}

\maketitle

\section*{Problem and Contribution}
The paper addresses the problem of \textbf{increasing the edge-connectivity of a simple $k$-edge-connected graph $G$ by one}, using the minimum number of edges from the complement of the graph $\bar{G}$. The key challenge basically is finding an efficient and minimal set of edges from the complement graph, $\bar{G}$, such that the resulting graph $G'$ becomes $(k+1)$-edge-connected.\\ When coming to the paper's main contribution, it is the identification of a dichotomy: depending on whether $\bar{G}$ contains a matching that covers all vertices of degree $k$ in $G$, the augmentation process follows two different algorithmic strategies, both achievable in \textbf{polynomial time}. In other words, we are dealing with a NP-hard problem and trying to come up with a solution in polynomial time.

\section*{Algorithmic Description}
The core idea of the algorithm is to decide between two augmentation strategies based on the structure of $\bar{G}$:

\begin{itemize}
    \item \textbf{Case 1:} If $\bar{G}$ contains a matching covering all vertices of degree $k$ in $G$, then this matching can be used directly as the augmenting edge set. The new graph $G \cup M$ becomes $(k+1)$-edge-connected.
    \item \textbf{Case 2:} If such a matching does not exist, the algorithm constructs a path-based augmenting system where paths of length 1 or 2 (max) are used to connect unmatched vertices. This system is initially found as a minimum-degree augmenting path system and then converted into an edge-connectivity augmenting one using the same number of edges.
\end{itemize}

\textbf{The inputs to the algorithm are:}
\begin{itemize}
    \item A simple graph $G$ that is $k$-edge-connected.
    \item Its complement $\bar{G}$.
\end{itemize}

\textbf{The output is:}
\begin{itemize}
    \item A set of edges from $\bar{G}$ that, when added to $G$, makes it $(k+1)$-edge-connected.
\end{itemize}

% \begin{figure}[h!]
% \centering
% \includegraphics[width=0.6\textwidth]{augmentation-diagram.png} % Replace with actual path if available
% \caption{High-level idea: complement matching leads to augmentation}
% \end{figure}

\section*{Comparison with Existing Approaches}
The traditional edge-connectivity augmentation problems make use of the addition of arbitrary edges. However, there are more efficient algorithmic-based approaches which we as Algorithms and Design students aim to look at.  What makes this paper novel is the restriction to edges only from the \textbf{complement graph}. This variation has not been studied as extensively and presents unique challenges due to its structural constraints. The algorithm leverages known techniques (like Edmonds' matching algorithm) but applies them in a constrained and direct way, which gives a solution that is efficient in theory and precise in application. 

\section*{Data Structures and Techniques}
\begin{itemize}
    \item Coming to the dataset itself, our idea is to work with more synthetic data sets, making graphs ourselves to make comparisions.
    \item \textbf{Matching Theory:} Specifically, Edmonds' Blossom Algorithm for finding maximum matching in general graphs.
    \item \textbf{Graph Complements:} Using $\bar{G}$ to select feasible augmenting edges.
    \item \textbf{Subgraph Induction:} Constructing $G_k$, the subgraph of vertices with degree $k$.
    \item \textbf{Path Systems:} Using paths of length 1 or at max 2 to construct efficient augmentation sets.
\end{itemize}

\section*{Implementation Outlook}

\begin{itemize}
    \item \textbf{A Matching Algorithms:} Implementing Edmonds' Blossom Algorithm might be a bit complex and requires careful data structure design. It makes use of some new concepts (e.g., alternating trees, union-find structures).
    \item \textbf{Graph Representation:} Maintaining both the original graph and its complement efficiently is essential, especially for large graphs.
    \item \textbf{Scalability:} While the algorithm is polynomial, its runtime may still be high for graphs with thousands of nodes. The paper itself is dealing with specific types of simple graphs.
    \item \textbf{Path Transformation:} The transformation from minimum-degree path systems to edge-connectivity path systems must preserve correctness and may involve intricate logic.
\end{itemize}

\section*{Implementation into Algorithm: A Basic Idea}

\begin{enumerate}
    \item \textbf{Input Representation} \\
    Accept a simple undirected graph $G = (V, E)$ as input. The graph can be represented using either an adjacency list or an adjacency matrix. Compute the edge-connectivity $k$ of the graph using standard graph algorithms.

    \item \textbf{Identify Critical Vertices} \\
    Identify the set $V_k$ of vertices in $G$ that have degree exactly $k$. 
    
    \item \textbf{Construct Complement Graph} \\
    Construct the complement graph $\bar{G} = (V, \bar{E})$ such that $\bar{E} = \{(u, v) \mid u, v \in V, u \neq v, (u, v) \notin E\}$. Focus on the induced subgraph $\bar{G}_k$ formed by the vertices in $V_k$.

    \item \textbf{Apply Edmonds’ Matching Algorithm} \\
    Use Edmonds’ Blossom Algorithm to compute a maximum matching $M$ in $\bar{G}_k$. This step is for identifying the largest set of disjoint pairs of vertices in $V_k$.

    \item \textbf{Case Analysis} \\
    \begin{itemize}
        \item \textbf{Case 1:} If the matching $M$ covers all vertices in $V_k$, add all edges from $M$ to $G$. The resulting graph becomes $(k + 1)$-edge-connected.
        \item \textbf{Case 2:} If the matching does \emph{not} cover all vertices in $V_k$, construct a path-based augmentation system using paths of length 1 or 2 to connect the unmatched vertices. Choose the minimal number of edges required to ensure increased edge-connectivity.
    \end{itemize}

    \item \textbf{Updating the graph and check} \\
    Add the selected edges (from the matching or path-based augmentation) to the original graph $G$. Optionally, recompute the edge-connectivity to verify the correctness of the augmentation.

\end{enumerate}

\end{document}
